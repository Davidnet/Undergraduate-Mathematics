\documentclass[11pt,twoside]{book}
\title{Undegraduate Mathematics}
\author{David Cardozo}

\usepackage{mathtools}
\usepackage{epigraph}
\usepackage{amsfonts}

\newcommand{\goesto}{\rightarrow}
\newcommand{\CC}{\mathbb{C}}
\newcommand{\abs}[1]{\left\| #1 \right\|}
\newcommand{\conj}[1]{\overline{#1}}
\newcommand{\setbraces}[1]{\left\lbrace #1 \right\rbrace}
\newcommand{\RR}{\mathbb{R}}

\begin{document}
\maketitle
\chapter{Preface}
This project started as a posible review of all topics revisited in an undergraduate math course.

\mainmatter
\chapter{Introduction to Mathematics}
First, a word about sets. These are the most primitive objects in mathematics. We use the following observation by Cantor

\epigraph{By an aggregate [set] we are to understand any collection into a whole M of definite and separate objects m of our intution or our thought. These objects we call the elements of M}{\textit{Cantor}}

We start with an informal definition of Natural numbers, recall that we can put this in stone observations, using the Peano axioms.

\chapter{Vector Calculus}

\subsection{Stokes Theorem}
Stokes'theorem relates the line integral of a vector field around a simple closed curve $C$ in $\RR^3$ to an integral over a surface $S$ for wic $C$ is the boundary.

\textbf{Stokes Theorem for Graph}
Consider $S$ that is the graph of a function $f(x,y)$ so that is parametrized by

\[
\begin{cases}
	x = u \\
	y = v \\
	z = f(u,v)
\end{cases}
\]

\textbf{Stokes' Theorem for Graphs} Let $S$ be the oriented surface by a $C^2$ function $z = f(x,y)$, where $(x,y) \in D$, a region to which Green's theorem applies, and let $\textbf{F}$ be a $C^1$ vector field on $S$. Then if $\partial S$ denotes the oriented boundary curves of $S$ as just defined, we have

\[
	\iint_S (\nabla \times \textbf{F}) \cdot d \textbf{S} = \int_{\partial S} \textbf{F} \cdot d \textbf{s}
\]


\textbf{Stokes' Theorem for Parametrized surfaces}

Let $S$ be an oriented surfaces defined by a one-to-one parametrization $\Phi :D \subset \RR^2 \goesto S$, where $D$ is a region to which Green's theorem applies. Let $\partial S$ denote the oriented boundary of $S$ and let $\textbf{F}$ be a $C^1$ vector field on $S$. Then
\[
	\iint_s (\nabla \times \textbf{F}) \cdot d \textbf{S} = \int_{\partial S} \textbf{F} \cdot d \textbf{s}
\]


\chapter{Linear Algebra}

A vector space $V$ over a filed $\mathbb{F}$ is an abelian group $(V,+)$, for which a binary product, $(a,v) \rightarrow av$, of $ \mathbb{F} \times V$ into $V$ is defined satisfying the following axioms for all $a,b \in \mathbb{F}$ and $u,v \in V$

\begin{itemize}
    \item $1v = v$
    \item $(ab)v = a(bv) $
    \item $(a + b)v = av + bv$ and $a(v+u) = av + au$
\end{itemize}

The elements of $V$ are usually referred as vectors; the elements of the underlying field as scalars.

Examples of vector spaces:

\begin{itemize}
	\item $\mathbb{F}^n $ the space of all F-valued n-tuples with addition and multiplication by scalars defined pointwise. To be able to differentiate between row and column spaces we will denote by the following $\mathbb{F}_c^n$ or $\mathbb{F}_r^n$.
	\item $M(n,m;\mathbb{F})$ are the default matrices with entries from F. 
\end{itemize}

Definition: A map $\phi: V \rightarrow W$ is linear if for all scalars $a,b$ and vectors $v_1, v_2$, we have
\begin{equation*}
	\phi(av_1 + b v_2) = a \phi(v_1) + b \phi(v_2)
\end{equation*}

A map $\phi$ is an isomorphism if it is both bijective and linear.

We observe that the relation of being isomorphic is an equivalence relation.

Definition: A (vector) subspace of a vector space $V$ is a subset that is closed under the operations of addition and multiplication by scalars inherited from $V$.
 
An interesting subspace of a vector space is to take $W_j$ be a family of subsets, then we have that the sum of subspaces, is the set

\[
\sum W_j = \bigcup_{J_1 \subset J} \left\lbrace v : v = \sum_{j \in J_1} v_j, v_j \in W_j \right\rbrace 
\]

\subsection{Quotient spaces}

A subspace $W$ of a vector space $V$ defines an equivalence relation in $V$

\[
	x \equiv y  \mod W
\]

This equivalence relation partitions $V$  into equivalence classes, called the cosets of $W$ in $V$. For $ x \in V$, the coset of $W$ that contains $x$ is the set $\tilde{x} = x + W$ the \textit{translate} of x by W.

Proposition 
\[
	\tilde{x} = x + W
\]

Consider an  arbitrary element of $b$ of $ \tilde{x} $,
\begin{align*}
	b \equiv x \mod W \\
	b - x = w_1
\end{align*}
For some $w_1$ in $W$, so that $b \in x + W$. The other side is similar.

We define the quotient space $V/W$ to be the space whose elements are the cosets of $W$ in $V$

Definition \textbf{Direct Sums}
If $V_1, \ldots, V_k$ are vector spaces over $\mathbb{F}$, the (formal) direct sum
\[
	\oplus_1^k V_j 
\]

is the set $ \lbrace (v_1,\ldots,v_k) : v_j \in V \rbrace $


\chapter{Complex Analysis}

The central objects are functions from the complex plane to itself
\[
	f: \mathbb{C} \goesto \mathbb{C}
\]

A more interesting anotation is that $f$ is differentiable in the complex sense. This condition is called holomorphicity, and it shapes all of complex analysis.

A function $f:\mathbb{C} \goesto \mathbb{C}$ is holomorphic at the point $z \in \mathbb{C}$ if the limit
\[
	\lim_{h \goesto 0} \frac{f(z+h)-f(z)}{h} \quad h \in \CC
\]

This encompasses a multiplicity of conditions: so to speak, one for each angle that $h$ can approach zero.

Our main goals is to observe the following properties.

\begin{itemize}
	\item Contour Integration. If $f$ is holomorphic in $\omega$, then for appropriate closed paths in $\omega$.
	\[
		\int_\gamma f(z) dz = 0
	\]
	\item Regularity. If $f$ is holomorphic, then $f$ is indefinitely differentiable
	\item Analytic continuation. If $f$ and $g$ are holomorphic functions in $\omega$ which are equal in an arbitrary small disc in $\omega$, then $f = g$ everywhere in $\omega$
\end{itemize}

Basic Properties

A complex number takes the form $z = x + iy$ where $x,y $ are real numbers and $i$ is an imaginray number that satisfies $i^2 = -1$, we denote $x = \operatorname{Re}(z), y = \operatorname{Im}(z) $, we observe that the real numbers are precisely those complex numbers for which the imaginary part is zero.
An important observation is the multiplication of two complex numbers:

\begin{align*}
	z_1 z_2 &= (x_1 + i y_1)(x_2 + i y_2) \\
			&= (x_1x_2 - y_1y_2) + i(x_1y_2 + y_1x_2)
\end{align*}

We observe that addition correspond naturally to the addition of two vectors, while multiplication consist of a rotation plus a dilatation, multiplication of $i$ is a rotation of $\frac{\pi}{2}$. We define, the absolute value of a complex number $z$ by
\[
	|z| = (x^2 + y^2)^{\frac{1}{2}}
\]

We observe then $|z|$ consists of the distance from the origin to the point $(x,y)$.

We observe that the triangle inequality holds

\[
	| z + w | <= |z| + |w| \quad \forall z,w \in \CC
\]

We observe the above since

\begin{align*}
| z + w |^2 &= (z + w)(\conj{z + w}) = (z+w)(\conj{z} + \conj{w}) \\
&= |z|^2 + w \conj{z} + z \conj{w} + |w|^2 \\
&= |z|^2 + w \conj{z} + \conj{w \conj{z}} + |w|^2 
\end{align*}
Now, we make the observation that 
\[
Re(a) = \frac{a + \conj{a}}{2}
\]
\begin{align*}
| z + w |^2 &= (z + w)(\conj{z + w}) = (z+w)(\conj{z} + \conj{w}) \\
			&= |z|^2 +2 \operatorname{Re}(w \conj{z}) + |w|^2 \\
			&\leq |z|^2 + 2 |w| |\conj{z} | + |w|^2 \\
			&= (|z| + |w|)^2
\end{align*}

The reverse triangle inequality is

\[
	||z|-|w|| \leq |z - w|
\]

and is proven as:
with the triangle inequality we have
\begin{align*}
	|z| + |w-z| \geq |z + w - z| = |w| \\
	|w| + |z-w| \geq |w + z - w | = |z| 
\end{align*}
and we observe then
\begin{align*}
	| w - z | \geq |w| - |z| \\
	| z - w | \geq |z| - |w|
\end{align*}
from absolute values we know that $ | w - z| = | z - w| $, and if $ t \geq a $ and $ t \geq -a$, implies $ t \geq |a| $, so that
\begin{align*}
	| z - w | \geq | |w| - |z| |
\end{align*}

We already have used $\conj{z}$ to denote the complex conjugate of $z$. we observe that

\begin{align*}
z\conj{z} = |z|^2 \quad \implies \frac{1}{z} = \frac{\conj{z}}{|z|^2}
\end{align*}

Any non-zero complex number $z$ can be written in polar form

\[
	z = re^{i \theta} 
\]
From here, we obtain the observation that if $z$ and $w$ are complex numbers, we have that their multiplication is:
\[
 zw = rse^{ i (\theta + \phi) }
\]
so multiplication by a complex number corresponds to a homothety in the plane.

\subsection{Convergence}

A sequence $\setbraces{z_1,z_2, \ldots} $ of complex numbers is said to converge to $w$ if
\[
	\lim_{n \goesto \infty} | z_n - w | = 0 
\]

Since absolute values in $\CC$ and Euclidean distances in the plane coincide, we see that $z_n$ converges to $w$ if and only if the corresponding sequence of points in ht complex plane converges to the point that correspond to $w$

In fact the sequence $\setbraces{z_n}$ converges to $w$ if and only if the sequence of real and imaginary parts of $z_n$ converge to the real and imaginary parts of $w$, respectively


\end{document}
